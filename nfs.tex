\subsection{6.4 NFS接続設定}

NFS(Network File System)を利用することで、マスターノードから他ノードやNAS上の共有フォルダへファイルアクセスが可能となり、データの共有・一元管理が容易になる。本節では、MacにおけるNFSの設定手順を示す。

\subsubsection*{NFS接続の事前準備}

NFS接続を行うために、まずMacの設定から「ファイル共有」と「画面共有」を有効にし、必要なユーザーの共有許可を設定しておく。

\begin{enumerate}
    \item Homebrewをインストールしておく(必要であれば)。
    \item ターミナルを開き、以下のコマンドで\texttt{/etc}ディレクトリへ移動する。
    \begin{verbatim}
    cd /etc
    \end{verbatim}
    \item \texttt{hosts}ファイルを編集して、対象ホストの名前解決を行えるようにする。
    \begin{verbatim}
    sudo nano hosts
    \end{verbatim}
    \item 以下のようにホスト情報を追記する。
    \begin{verbatim}
    192.168.100.200 SAIL-S1
    192.168.100.10  SAIL-10
    192.168.100.8   SAIL-08
    \end{verbatim}
\end{enumerate}

\subsubsection*{自動マウントの設定}

\begin{enumerate}
    \item マウントポイントを作成する。
    \begin{verbatim}
    sudo mkdir /Volumes/nfs_share
    \end{verbatim}
    \item \texttt{auto\_nfs}ファイルを新規作成または編集し、以下のように記述する。
    \begin{verbatim}
    sudo nano /etc/auto_nfs
    \end{verbatim}
    \begin{verbatim}
    /Volumes/nfs_share -fstype=nfs,nosuid 192.168.100.200:/volume1/For_Clustrer
    \end{verbatim}
    \item \texttt{auto\_master}に\texttt{auto\_nfs}を読み込むよう記述を追加する。
    \begin{verbatim}
    sudo nano /etc/auto_master
    \end{verbatim}
    ファイル末尾に以下の一行を追加:
    \begin{verbatim}
    +auto_nfs
    \end{verbatim}
    \item 変更を反映させるため、マウント設定を更新する。
    \begin{verbatim}
    sudo automount -vc
    \end{verbatim}
\end{enumerate}

\subsubsection*{手動マウントの手順(確認用)}

自動マウントの代わりに手動でマウントする場合は、以下のコマンドを使用する。

\begin{verbatim}
sudo mount -t nfs 192.168.100.200:/volume1/For_Cluster /Users/ccadmin/nfs_share
\end{verbatim}

\subsubsection*{マウント確認}

ホームディレクトリに移動し、マウントされた共有フォルダが見えているか確認する。

\begin{verbatim}
cd
ls nfs_share
\end{verbatim}

この設定により、NFSを通じたマスターノードからのファイル共有環境が整い、複数ノード間でのデータアクセスの一貫性が確保される。
